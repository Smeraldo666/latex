%%%%%%%%%%%%%%%%%%%%%%%%%%%%%%%%%%%%%%%%%%%%%%%%%%%%%%%%%%%%
% 数学分析笔记
% 作者: Smeraldo
% 日期: 2025.3.14
%%%%%%%%%%%%%%%%%%%%%%%%%%%%%%%%%%%%%%%%%%%%%%%%%%%%%%%%%%%%

\documentclass[12pt, UTF8, AutoFakeBold]{ctexart} % 文章文档类,12pt 字号

% ========== 基本包 ==========
\usepackage{amsmath, amssymb, amsthm} % 数学符号、定理环境
\usepackage{geometry} % 控制页面布局
\geometry{a4paper, margin=1in} % 设置 A4 纸张,1 英寸边距
\usepackage{graphicx} % 插入图片
\usepackage{xcolor} % 颜色支持
\usepackage{tcolorbox} % 重点内容高亮
\usepackage{hyperref} % 超链接支持
\hypersetup{
    colorlinks=true,
    linkcolor=blue,
    urlcolor=magenta
}

% ========== 定理、定义、引理 ==========
\newtheorem{theorem}{定理}[section]
\newtheorem{definition}{定义}[section]
\newtheorem{lemma}{引理}[section]
\newtheorem{example}{例}[section]

% ========== 常用数学符号定义 ==========
\newcommand{\R}{\mathbb{R}} % 实数集
\newcommand{\dx}{\,dx} % 微分符号
\newcommand{\limn}{\lim_{n\to\infty}} % 极限格式

% ========== 目录 ==========
\begin{document}
\tableofcontents % 生成目录
\clearpage

\section{数项级数}

\subsection{级数的敛散性}
\begin{definition}
    给定一个数列$\{u_n\}$,对它的各项依次用“$+$”号连接起来的表达式
    \[
        u_1 + u_2 + \dots + u_n + \dots
    \]
    称为\textbf{\kaishu 常数项无穷级数}或\textbf{\kaishu 数项级数}(也常简称\textbf{\kaishu 级数}),
    其中$u_n$称为数项级数的\textbf{\kaishu 通项}或\textbf{\kaishu 一般项}.\\
    数项级数的前n项之和,记为
    \[
        S_n = \sum_{k=1}^{n}u_k = u_1 + u_2 + \dots + u_n,
    \]
    称它为数项级数的\textbf{\kaishu 第n个部分和},也简称\textbf{\kaishu 部分和}.
\end{definition}

\begin{definition}
    若数项级数的部分和数列$\{S_n\}$收敛于S(即$\lim\limits_{n \to \infty}S_n = S$),则称数项级数\textbf{\kaishu 收敛},称S为数项级数的\textbf{\kaishu 和},记作
    \[
        S = u_1 + u_2 + \dots + u_n + \dots \;\; \text{或} \;\; S = \sum u_n
    \]
    若$\{S_n\}$是发散数列,则称数项级数\textbf{\kaishu 发散}.
\end{definition}

\begin{example}
    讨论等比级数(几何级数)
    \[
        a + aq + aq^2 + \dots + aq^n + \dots
    \]
    的敛散性($a \neq 0$)\\
    \textbf{解} $q \neq 1$ 时,级数的第n个部分和
    \[
        S_n = a + aq + \dots + aq^{n-1} = a \cdot \frac{1 - q^n}{1 - q} 
    \]
    因此,\\
    \indent (i)当 $\lvert q \rvert < 1$ 时,
    $\lim\limits_{n \to \infty} = \lim\limits_{n \to \infty} a \cdot \frac{1 - q^n}{1 - q} = \frac{a}{1 - q}$. 
    此时级数收敛,其和为$\frac{a}{1 - q}$.\\
    \indent (ii)当 $\left\lvert q \right\rvert < 1$ 时,
    $\lim\limits_{n \to \infty} = \infty$,级数发散.\\
    \indent (iii)当 $q = 1$ 时,$S_n = na$,级数发散.\\
    \indent 当 $q = -1$ 时,$S_{2k} = 0, S_{2k+1} = a, k = 0, 1, 2, \dots,$级数发散.\\
    \indent 总之,$\lvert q \rvert < 1$时,级数收敛;$\lvert q \rvert \geqslant 1$时,级数发散.
\end{example}

\begin{tcolorbox}[colback=yellow!20, colframe=red!80!black]
    \begin{theorem}
        (级数收敛的柯西准则)

        级数收敛的充要条件是:任给正数$\varepsilon$,总存在正整数N,使得当m>N以及对任意的正整数p,都有
        \begin{equation}
            |u_{m+1} + u_{m+2} + \dots + u_{m+p}| < \varepsilon. \tag{1}
        \end{equation}

        级数发散的充要条件是:存在某正数$\varepsilon_0$,对任何正整数N,总存在正整数$m_0(>N)$和$p_0$,有
        \begin{equation}
            |u_{m+1} + u_{m+2} + \dots + u_{m+p}| \geq  \varepsilon_0. \tag{2}
        \end{equation}

        \textbf{推论} 若级数收敛,则
        \[
            \lim\limits_{n \to \infty} u_n = 0.
        \]

            当一个级数$\sum\limits_{n = 1}^{\infty}u_n$的一般项$u_n$不收敛于零时,由推论可知该级数发散.
    \end{theorem}
\end{tcolorbox}

\begin{example}
    证明\textbf{\kaishu 调和级数}
    \[
        1 + \frac{1}{2} + \frac{1}{3} + \dots + \frac{1}{n} + \dots
    \]
    是发散的.\\
    \textbf{证} 由
    \[
        \limn u_n = \limn \frac{1}{n} = 0,
    \]
    无法用推论推出调和级数发散.但令p=m时,有
    \[
    \begin{aligned}
        |u_{m+1} + u_{m+2} + \dots + u_{2m}| &= \left|\frac{1}{m + 1} + \frac{1}{m + 2} + \dots + \frac{1}{2m}\right|\\
        &\geq \left|\frac{1}{2m} + \frac{1}{2m} + \dots + \frac{1}{2m}\right|\\
        &= \frac{1}{2},
    \end{aligned}
    \]
    取$\varepsilon_0 = \frac{1}{2}$, 对任何正整数N,只要m>N和p=m就有(2)式成立,所以调和级数是发散的.
\end{example}

\begin{theorem}
    若级数$\sum u_n$与$\sum v_n$都收敛,则对任意常数c,d,级数$\sum (cu_n+dv_n)$亦收敛,且
    \[
        \sum (cu_n+dv_n) = c\sum u_n+d\sum v_n.
    \]
\end{theorem}

\begin{theorem}
    去掉、增加或改变级数的有限个项并不改变级数的敛散性.
\end{theorem}

\begin{theorem}
    在收敛级数的项中任意加括号,既不改变级数的收敛性,也不改变它的和.
\end{theorem}

\subsection{正项级数}
\subsubsection{正项级数收敛性的一般判别原则}
\begin{theorem}[比较原则]
    设$\sum u_n$和$\sum v_n$是两个正项级数,如果存在某正数N,对一切n>N都有
    \[
        u_n \leq v_n
    \]
    则

    (i)若级数$\sum v_n$收敛,则级数$\sum u_n$也收敛;

    (ii)若级数$\sum u_n$发散,则级数$\sum v_n$也发散.\\
    \textbf{推论} 设
    \begin{gather}
        u_1 + u_2 + \dots + u_n + \dots,\\
        v_1 + v_2 + \dots + v_n + \dots
    \end{gather}
    是两个正项级数,若
    \[
        \lim_{n \to \infty}\frac{u_n}{v_n} = l
    \]
    则

    (i)当$0<l<+\infty$时,级数(1)(2)同时收敛或同时发散;

    (ii)当$l=0$且级数(2)收敛时,级数(1)也收敛;

    (iii)当$l=+\infty$且级数(2)发散时,级数(1)也发散.
\end{theorem}

\subsubsection{比式判别法和根式判别法}
\begin{theorem}[拉朗贝尔判别法,或称比式判别法]
    设$\sum u_n$为正项级数,且存在某正整数$N_0$及常数q($0<q<1$).

    (i)若对一切$n>N_0$,成立不等式
    \[
        \frac{u_{n+1}}{u_n} \leq q,
    \]
    则级数$\sum u_n$收敛.

    (ii)若对一切$n>N_0$,成立不等式
    \[
        \frac{u_{n+1}}{u_n} \geq 1,
    \]
    则级数$\sum u_n$发散.\\
    \textbf{推论$1$}(比式判别法的极限形式)若$\sum u_n$为正项级数,且
    \begin{gather}
        \limn\frac{u_{n+1}}{u_n}=q,
    \end{gather}
    则

    (i)当$q<1$时,级数$\sum u_n$收敛;

    (ii)当$q>1$或$q=+\infty$时,级数$\sum u_n$发散.

    若$q=1$,这时用比式判别法不能对级数的敛散性作出判断,因为它可能是收敛的,也可能是发散的.\\
    若某级数的($3$)式的极限不存在,则可应用上、下极限来判别.\\
    \textbf{推论$2$}设$\sum u_n$为正项级数.

    (i)若$\varlimsup\limits_{n\to\infty}\frac{u_{n+1}}{u_n}=q<1$,则级数收敛;

    (ii)若$\varliminf\limits_{n\to\infty}\frac{u_{n+1}}{u_n}=q>1$,则级数发散.
\end{theorem}
\begin{theorem}[柯西判别法,或称根式判别法]
    设$\sum u_n$为正项级数,且存在某正整数$N_0$及正常数l,

    (i)若对一切$n>N_0$,成立不等式
    \[
        \sqrt[n]{u_n} \leq l<1,
    \]
    则级数$\sum u_n$收敛;

    (ii)若对一切$n>N_0$,成立不等式
    \[
        \sqrt[n]{u_n} \geq 1,
    \]
    则级数$\sum u_n$发散.\\
    \textbf{推论$1$}(根式判别法的极限形式)若$\sum u_n$为正项级数,且
    \begin{gather}
        \limn\sqrt[n]{u_n} = l,
    \end{gather}
    则

    (i)当$l<1$时,级数$\sum u_n$收敛;

    (ii)当$l>1$时,级数$\sum u_n$发散.\\
    若($4$)式的极限不存在,则可根据根式$\sqrt[n]{u_n}$的上极限来判断.\\
    \textbf{推论$2$}设$\sum u_n$为正项级数,且
    \[
        \varlimsup\limits_{n\to\infty}\sqrt[n]{u_n}=l,
    \]
    则当

    (i)$l<1$时级数收敛;

    (ii)$l>1$时级数发散.
\end{theorem}
若
\[
    \limn\frac{u_{n+1}}{u_n}=q,
\]
则必有
\[
    \limn\sqrt[n]{u_n} = q.
\]
这说明凡能由比式判别法鉴别收敛性的级数,也能由根式判别法来判断.

\subsubsection{积分判别法}
\begin{theorem}
    设f为$[1,+\infty)$上的减函数,则级数$\sum\limits_{n=1}^{\infty}f(n)$收敛的充分必要条件是
    反常积分$\int_{1}^{+\infty}f(x)\, dx$收敛.
\end{theorem}

\subsection{一般项级数}
\end{document}